\documentclass[a4paper,10pt]{article}
\usepackage[utf8]{inputenc}
\usepackage{fancyhdr, float, graphicx, caption}
\usepackage{amsmath}
\usepackage[margin=1in]{geometry}
\usepackage{multicol}

\pagestyle{fancy}
\renewcommand{\figurename}{Figura}
\renewcommand\abstractname{\textit{Abstract}}

\fancyhf{}
\fancyhead[LE,RO]{\textit{Estructuras de datos y algoritmos II}}
\fancyfoot[RE,CO]{\thepage}

\title{
	%Logo UNR
	\begin{figure}[!h]
		\centering
		\includegraphics[scale=1]{unr.png}
		\label{}
	\end{figure}
	% Pie Logo
	\normalsize
		\textsc{Universidad Nacional de Rosario}\\	
		\textsc{Facultad de Ciencias Exactas, Ingeniería y Agrimensura}\\
		\textit{Licenciatura en Ciencias de la Computación}\\
		\textit{Estructuras de datos y algoritmos II}\\
	% Título
	\vspace{30pt}
	\hrule{}
	\vspace{15pt}
	\huge
		\textbf{Especificación de costos}\\
	\vspace{15pt}
	\hrule{}
	\vspace{30pt}
	% Alumnos/docentes
	\begin{multicols}{2}
	\raggedright
		\large
			\textbf{Alumnos:}\\
		\normalsize
			CRESPO, Lisandro (C-6165/4) \\
			MISTA, Agust\'in (M-6105/1) \\
	\raggedleft
		\large
			\textbf{Docentes:}\\
		\normalsize
			JASKELIOFF, Mauro\\
			RABASEDAS, Juan Manuel\\
			SIMICH, Eugenia\\
	\end{multicols}
}

\begin{document}
\date{1 de Junio de 2015}
\maketitle

\pagebreak
\part*{Implementación con listas}
	\section*{\Large filterS}
		Para implementar la función \texttt{filterS}, consideramos la función \texttt{filter} presente en el preludio,
		y paralelizamos el llamado recursivo para mejorar el rendimiento si los predicados que \texttt{filterS} evalúa
		son costosos de calcular.
		Luego podemos considerar a \texttt{filterS} como la siguiente recurrencia:
		
		\begin{equation*}
			T\left( n \right) = T\left( n-1 \right) + f\left( n \right)
		\end{equation*}		
		
		Donde $f(n)$ es el costo de evaluar cada predicado, además del costo de las comparaciones, que consideramos constantes.
		Puede verse que en esta implementación, el paralelizar las operaciones no mejora el problema de tener que recorrer todo
		el arreglo de forma secuencial. Resolviendo la recurrencia tenemos entonces:
		
		\begin{equation*}
			W\left( filterS \oplus s \right) \in O\left( \vert s \vert + \sum_{i=0}^{\vert s\vert -1} W \left( f\left( i\right) \right) \right)
		\end{equation*}		
		
		\begin{equation*}
			S\left( filterS \oplus s \right) \in O\left( \vert s \vert + \max_{i=0}^{\vert s\vert -1}\left( S \left( f\left( i\right) \right) \right)\right)
		\end{equation*}		
	
	
		Finalmente, si consideramos que $f(n) \in O(1)$ resulta:
		
		\begin{equation*}
			W\left( filterS \oplus s \right) \in O\left( n \right)
		\end{equation*}		
		
		\begin{equation*}
			S\left( filterS \oplus s \right) \in O\left( n \right)
		\end{equation*}		
		 	
	\section*{\Large showtS}
		Para el caso de \texttt{showtS}, la implementación mediante listas es poco eficiente dado que para poder partir la lista
		en dos mitades en el caso de que existan dos o más elementos, se necesita conocer el tamaño de la misma, lo cual resulta en un
		coste lineal tanto para el trabajo como para la profundidad. Por lo tanto:
	
		\begin{equation*}
			W\left( showtS \oplus s \right) \in O\left( n \right)
		\end{equation*}		
		
		\begin{equation*}
			S\left( showtS \oplus s \right) \in O\left( n \right)
		\end{equation*}	
	
	\section*{\Large redusceS}
	
	
	
	\section*{\Large scanS}
	
	
	
	
	
\pagebreak	
\part*{Implementación con arreglos persistentes}

	\section*{\Large filterS}
	
	
	
	\section*{\Large showtS}
	
	
	
	\section*{\Large redusceS}
	
	
	
	\section*{\Large scanS}


\pagebreak

Bla Bla Bla...

\vspace{\fill}
\begin{multicols}{2}
	\hrule
	\vspace{5pt}
	CRESPO, Lisandro \\
	\linebreak

	\hrule
	\vspace{5pt}
	MISTA, Agustín \\
\end{multicols}

\end{document}
